\documentclass[]{article}
\usepackage{lmodern}
\usepackage{amssymb,amsmath}
\usepackage{ifxetex,ifluatex}
\usepackage{fixltx2e} % provides \textsubscript
\ifnum 0\ifxetex 1\fi\ifluatex 1\fi=0 % if pdftex
  \usepackage[T1]{fontenc}
  \usepackage[utf8]{inputenc}
\else % if luatex or xelatex
  \ifxetex
    \usepackage{mathspec}
  \else
    \usepackage{fontspec}
  \fi
  \defaultfontfeatures{Ligatures=TeX,Scale=MatchLowercase}
\fi
% use upquote if available, for straight quotes in verbatim environments
\IfFileExists{upquote.sty}{\usepackage{upquote}}{}
% use microtype if available
\IfFileExists{microtype.sty}{%
\usepackage{microtype}
\UseMicrotypeSet[protrusion]{basicmath} % disable protrusion for tt fonts
}{}
\usepackage[margin=1in]{geometry}
\usepackage{hyperref}
\hypersetup{unicode=true,
            pdftitle={Grafico de Barras},
            pdfauthor={Luíza de Oliveira Machado Pinto},
            pdfborder={0 0 0},
            breaklinks=true}
\urlstyle{same}  % don't use monospace font for urls
\usepackage{color}
\usepackage{fancyvrb}
\newcommand{\VerbBar}{|}
\newcommand{\VERB}{\Verb[commandchars=\\\{\}]}
\DefineVerbatimEnvironment{Highlighting}{Verbatim}{commandchars=\\\{\}}
% Add ',fontsize=\small' for more characters per line
\usepackage{framed}
\definecolor{shadecolor}{RGB}{248,248,248}
\newenvironment{Shaded}{\begin{snugshade}}{\end{snugshade}}
\newcommand{\AlertTok}[1]{\textcolor[rgb]{0.94,0.16,0.16}{#1}}
\newcommand{\AnnotationTok}[1]{\textcolor[rgb]{0.56,0.35,0.01}{\textbf{\textit{#1}}}}
\newcommand{\AttributeTok}[1]{\textcolor[rgb]{0.77,0.63,0.00}{#1}}
\newcommand{\BaseNTok}[1]{\textcolor[rgb]{0.00,0.00,0.81}{#1}}
\newcommand{\BuiltInTok}[1]{#1}
\newcommand{\CharTok}[1]{\textcolor[rgb]{0.31,0.60,0.02}{#1}}
\newcommand{\CommentTok}[1]{\textcolor[rgb]{0.56,0.35,0.01}{\textit{#1}}}
\newcommand{\CommentVarTok}[1]{\textcolor[rgb]{0.56,0.35,0.01}{\textbf{\textit{#1}}}}
\newcommand{\ConstantTok}[1]{\textcolor[rgb]{0.00,0.00,0.00}{#1}}
\newcommand{\ControlFlowTok}[1]{\textcolor[rgb]{0.13,0.29,0.53}{\textbf{#1}}}
\newcommand{\DataTypeTok}[1]{\textcolor[rgb]{0.13,0.29,0.53}{#1}}
\newcommand{\DecValTok}[1]{\textcolor[rgb]{0.00,0.00,0.81}{#1}}
\newcommand{\DocumentationTok}[1]{\textcolor[rgb]{0.56,0.35,0.01}{\textbf{\textit{#1}}}}
\newcommand{\ErrorTok}[1]{\textcolor[rgb]{0.64,0.00,0.00}{\textbf{#1}}}
\newcommand{\ExtensionTok}[1]{#1}
\newcommand{\FloatTok}[1]{\textcolor[rgb]{0.00,0.00,0.81}{#1}}
\newcommand{\FunctionTok}[1]{\textcolor[rgb]{0.00,0.00,0.00}{#1}}
\newcommand{\ImportTok}[1]{#1}
\newcommand{\InformationTok}[1]{\textcolor[rgb]{0.56,0.35,0.01}{\textbf{\textit{#1}}}}
\newcommand{\KeywordTok}[1]{\textcolor[rgb]{0.13,0.29,0.53}{\textbf{#1}}}
\newcommand{\NormalTok}[1]{#1}
\newcommand{\OperatorTok}[1]{\textcolor[rgb]{0.81,0.36,0.00}{\textbf{#1}}}
\newcommand{\OtherTok}[1]{\textcolor[rgb]{0.56,0.35,0.01}{#1}}
\newcommand{\PreprocessorTok}[1]{\textcolor[rgb]{0.56,0.35,0.01}{\textit{#1}}}
\newcommand{\RegionMarkerTok}[1]{#1}
\newcommand{\SpecialCharTok}[1]{\textcolor[rgb]{0.00,0.00,0.00}{#1}}
\newcommand{\SpecialStringTok}[1]{\textcolor[rgb]{0.31,0.60,0.02}{#1}}
\newcommand{\StringTok}[1]{\textcolor[rgb]{0.31,0.60,0.02}{#1}}
\newcommand{\VariableTok}[1]{\textcolor[rgb]{0.00,0.00,0.00}{#1}}
\newcommand{\VerbatimStringTok}[1]{\textcolor[rgb]{0.31,0.60,0.02}{#1}}
\newcommand{\WarningTok}[1]{\textcolor[rgb]{0.56,0.35,0.01}{\textbf{\textit{#1}}}}
\usepackage{graphicx,grffile}
\makeatletter
\def\maxwidth{\ifdim\Gin@nat@width>\linewidth\linewidth\else\Gin@nat@width\fi}
\def\maxheight{\ifdim\Gin@nat@height>\textheight\textheight\else\Gin@nat@height\fi}
\makeatother
% Scale images if necessary, so that they will not overflow the page
% margins by default, and it is still possible to overwrite the defaults
% using explicit options in \includegraphics[width, height, ...]{}
\setkeys{Gin}{width=\maxwidth,height=\maxheight,keepaspectratio}
\IfFileExists{parskip.sty}{%
\usepackage{parskip}
}{% else
\setlength{\parindent}{0pt}
\setlength{\parskip}{6pt plus 2pt minus 1pt}
}
\setlength{\emergencystretch}{3em}  % prevent overfull lines
\providecommand{\tightlist}{%
  \setlength{\itemsep}{0pt}\setlength{\parskip}{0pt}}
\setcounter{secnumdepth}{0}
% Redefines (sub)paragraphs to behave more like sections
\ifx\paragraph\undefined\else
\let\oldparagraph\paragraph
\renewcommand{\paragraph}[1]{\oldparagraph{#1}\mbox{}}
\fi
\ifx\subparagraph\undefined\else
\let\oldsubparagraph\subparagraph
\renewcommand{\subparagraph}[1]{\oldsubparagraph{#1}\mbox{}}
\fi

%%% Use protect on footnotes to avoid problems with footnotes in titles
\let\rmarkdownfootnote\footnote%
\def\footnote{\protect\rmarkdownfootnote}

%%% Change title format to be more compact
\usepackage{titling}

% Create subtitle command for use in maketitle
\providecommand{\subtitle}[1]{
  \posttitle{
    \begin{center}\large#1\end{center}
    }
}

\setlength{\droptitle}{-2em}

  \title{Grafico de Barras}
    \pretitle{\vspace{\droptitle}\centering\huge}
  \posttitle{\par}
    \author{Luíza de Oliveira Machado Pinto}
    \preauthor{\centering\large\emph}
  \postauthor{\par}
      \predate{\centering\large\emph}
  \postdate{\par}
    \date{Estatística é com R}


\begin{document}
\maketitle

\hypertarget{roteiro}{%
\subsection{Roteiro}\label{roteiro}}

Sejam bem-vindos ao Estatística é com R! No video de hoje, iremos
estudar como se faz um \textbf{gráfico de barras}.O gráfico de barras é
um gráfico com barras retangulares e comprimento proporcional aos
valores que ele representa.

Hoje vamos usar um exemplo, utilizando o nivel de escolaridade de um
determinado grupo de alunos.

\emph{Podemos observar que o objeto alunos foi criado , e logo em
seguida geramos um gráfico de barras simples}

\begin{Shaded}
\begin{Highlighting}[]
\NormalTok{alunos <-}\StringTok{ }\KeywordTok{c}\NormalTok{(}\DecValTok{1000}\NormalTok{, }\DecValTok{650}\NormalTok{, }\DecValTok{250}\NormalTok{)}
\KeywordTok{barplot}\NormalTok{(alunos)}
\end{Highlighting}
\end{Shaded}

\includegraphics{roteiro_barras_files/figure-latex/Exemplo-1.pdf}

\emph{Agora vamos acrescentar informações no gráfico, criamos um objeto
chamado ``escolaridade'', onde alocaremos os níveis de ensino. O comando
``names.arg = escolaridade'', gera um gráfico cujo cada barra será
nomeada. Logo após utilizamos o comando main , para criar o título do
nosso gráfico, lembre-se que o título deverá estar sempre entre aspas.
Em seguida nomeamos o eixo Y.}

\begin{Shaded}
\begin{Highlighting}[]
\NormalTok{escolaridade <-}\StringTok{ }\KeywordTok{c}\NormalTok{(}\StringTok{"Fundamental"}\NormalTok{, }\StringTok{"Médio"}\NormalTok{, }\StringTok{"Superior"}\NormalTok{)}
\KeywordTok{barplot}\NormalTok{(alunos, }\DataTypeTok{names.arg =}\NormalTok{ escolaridade, }\DataTypeTok{main=} \StringTok{"Grau de Escolaridade"}\NormalTok{,}
        \DataTypeTok{ylab =} \StringTok{"Quantidade de alunos"}\NormalTok{)}
\end{Highlighting}
\end{Shaded}

\includegraphics{roteiro_barras_files/figure-latex/Bar01-1.pdf}

\emph{Para criar um subtítulo utilize o comando sub.}

\begin{Shaded}
\begin{Highlighting}[]
\KeywordTok{barplot}\NormalTok{(alunos, }\DataTypeTok{names.arg =}\NormalTok{ escolaridade, }\DataTypeTok{main =} \StringTok{"Grau de Escolaridade"}\NormalTok{, }
        \DataTypeTok{ylab =} \StringTok{"Quantidade de alunos"}\NormalTok{, }\DataTypeTok{sub =} \StringTok{"Dados fictícios"}\NormalTok{)}
\end{Highlighting}
\end{Shaded}

\includegraphics{roteiro_barras_files/figure-latex/Bar02-1.pdf}

\emph{Podemos mudar o tamanho da fonte, do título, subtítulo e dos
eixos. Colocamos o comando cex.main (para o título), cex.lab(para o
titulo dos eixos), cex.axis(para a fonte dos números do eixo y) e
cex.sub (para o subtítulo).}

\begin{Shaded}
\begin{Highlighting}[]
\CommentTok{# Para o título}

\KeywordTok{barplot}\NormalTok{(alunos, }\DataTypeTok{names.arg =}\NormalTok{ escolaridade, }\DataTypeTok{main =} \StringTok{"Grau de Escolaridade"}\NormalTok{, }
        \DataTypeTok{ylab =} \StringTok{"Quantidade de alunos"}\NormalTok{, }\DataTypeTok{sub =} \StringTok{"Dados fictícios"}\NormalTok{, }\DataTypeTok{cex.main =} \DecValTok{2}\NormalTok{)}
\end{Highlighting}
\end{Shaded}

\includegraphics{roteiro_barras_files/figure-latex/Bar03-1.pdf}

\begin{Shaded}
\begin{Highlighting}[]
\CommentTok{# Para os eixos}

\KeywordTok{barplot}\NormalTok{(alunos, }\DataTypeTok{names.arg =}\NormalTok{ escolaridade, }\DataTypeTok{main =} \StringTok{"Grau de Escolaridade"}\NormalTok{, }
        \DataTypeTok{ylab =} \StringTok{"Quantidade de alunos"}\NormalTok{, }\DataTypeTok{sub =} \StringTok{"Dados fictícios"}\NormalTok{, }\DataTypeTok{cex.lab =} \FloatTok{1.5}\NormalTok{)}
\end{Highlighting}
\end{Shaded}

\includegraphics{roteiro_barras_files/figure-latex/Bar03-2.pdf}

\begin{Shaded}
\begin{Highlighting}[]
\CommentTok{# Para o número do eixo horizontal}

\KeywordTok{barplot}\NormalTok{(alunos, }\DataTypeTok{names.arg =}\NormalTok{ escolaridade, }\DataTypeTok{main=} \StringTok{"Grau de Escolaridade"}\NormalTok{, }
        \DataTypeTok{ylab =} \StringTok{"Quantidade de alunos"}\NormalTok{, }\DataTypeTok{sub =} \StringTok{"Dados fictícios"}\NormalTok{, }\DataTypeTok{cex.axis =} \DecValTok{2}\NormalTok{)}
\end{Highlighting}
\end{Shaded}

\includegraphics{roteiro_barras_files/figure-latex/Bar03-3.pdf}

\begin{Shaded}
\begin{Highlighting}[]
\CommentTok{# Para o subtítulo}

\KeywordTok{barplot}\NormalTok{(alunos, }\DataTypeTok{names.arg =}\NormalTok{ escolaridade, }\DataTypeTok{main =} \StringTok{"Grau de Escolaridade"}\NormalTok{, }
        \DataTypeTok{ylab =} \StringTok{"Quantidade de alunos"}\NormalTok{, }\DataTypeTok{sub =} \StringTok{"Dados fictícios"}\NormalTok{, }\DataTypeTok{cex.sub =} \DecValTok{2}\NormalTok{)}
\end{Highlighting}
\end{Shaded}

\includegraphics{roteiro_barras_files/figure-latex/Bar03-4.pdf}

\emph{Para eliminar os eixos use o comando axes = FALSE ou F}

\begin{Shaded}
\begin{Highlighting}[]
\KeywordTok{barplot}\NormalTok{(alunos, }\DataTypeTok{names.arg =}\NormalTok{ escolaridade, }\DataTypeTok{main =} \StringTok{"Grau de Escolaridade"}\NormalTok{, }
        \DataTypeTok{ylab =} \StringTok{"Quantidade de alunos"}\NormalTok{, }\DataTypeTok{axes =}\NormalTok{ F)}
\end{Highlighting}
\end{Shaded}

\includegraphics{roteiro_barras_files/figure-latex/Bar04-1.pdf}

\emph{Para inverter a posição das barras, ou seja, criar um gráfico de
barras horizontal, trocamos o eixo y pelo eixo x, e dizemos horiz = T}

\begin{Shaded}
\begin{Highlighting}[]
\KeywordTok{barplot}\NormalTok{(alunos, }\DataTypeTok{names.arg =}\NormalTok{ escolaridade, }\DataTypeTok{main =} \StringTok{"Grau de Escolaridade"}\NormalTok{, }
        \DataTypeTok{xlab =} \StringTok{"Quantidade de alunos"}\NormalTok{, }\DataTypeTok{sub =} \StringTok{"Dados fictícios"}\NormalTok{, }\DataTypeTok{horiz =}\NormalTok{ T)}
\end{Highlighting}
\end{Shaded}

\includegraphics{roteiro_barras_files/figure-latex/Bar05-1.pdf}

\emph{A função density acrescenta o sombreamento as barras}

\begin{Shaded}
\begin{Highlighting}[]
\KeywordTok{barplot}\NormalTok{(alunos, }\DataTypeTok{names.arg =}\NormalTok{ escolaridade, }\DataTypeTok{main =} \StringTok{"Grau de Escolaridade"}\NormalTok{, }
        \DataTypeTok{xlab =} \StringTok{"Quantidade de alunos"}\NormalTok{, }\DataTypeTok{sub =} \StringTok{"Dados fictícios"}\NormalTok{, }\DataTypeTok{horiz =}\NormalTok{ T)}
\end{Highlighting}
\end{Shaded}

\includegraphics{roteiro_barras_files/figure-latex/Bar06-1.pdf}

\emph{O comando angle muda os angulos das linhas que preenchem as
barras}

\begin{Shaded}
\begin{Highlighting}[]
\KeywordTok{barplot}\NormalTok{(alunos, }\DataTypeTok{names.arg =}\NormalTok{ escolaridade, }\DataTypeTok{main =} \StringTok{"Grau de Escolaridade"}\NormalTok{, }
        \DataTypeTok{ylab =} \StringTok{"Quantidade de alunos"}\NormalTok{, }\DataTypeTok{sub =} \StringTok{"Dados fictícios"}\NormalTok{, }\DataTypeTok{density =} \DecValTok{20}\NormalTok{, }\DataTypeTok{angle =} \DecValTok{180}\NormalTok{)}
\end{Highlighting}
\end{Shaded}

\includegraphics{roteiro_barras_files/figure-latex/Bar07-1.pdf}

\emph{Este vídeo é da série sobre Visualização de dados utilizando
comandos do R Básico. Não se esqueça de curtir e ativar as notificações
do canal. Até a próxima pessoal!}


\end{document}
